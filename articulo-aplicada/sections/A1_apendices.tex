\appendix

\section{Esquema de Base de Datos Relacional}
\label{app:db_schema}

El modelo de datos de GESCOMPH se ha diseñado siguiendo la tercera forma normal (3NF) para garantizar la integridad referencial. A continuación se describen las entidades principales y sus relaciones:

\begin{description}
    \item[Contract (Contrato):] Entidad central del sistema.
    \begin{itemize}
        \item \texttt{Id (PK, int)}: Identificador único.
        \item \texttt{PersonId (FK, int)}: Referencia al arrendatario.
        \item \texttt{TotalBaseRentAgreed (decimal)}: Valor del canon pactado.
        \item \texttt{Active (bool)}: Estado del contrato.
        \item \texttt{StartDate / EndDate (datetime)}: Vigencia.
    \end{itemize}

    \item[Establishment (Establecimiento):] Unidad física arrendable.
    \begin{itemize}
        \item \texttt{Id (PK, int)}: Identificador único.
        \item \texttt{PlazaId (FK, int)}: Ubicación macro.
        \item \texttt{RentValueBase (decimal)}: Valor base del canon.
        \item \texttt{Active (bool)}: Disponibilidad.
    \end{itemize}

    \item[PremisesLeased (Locales Arrendados):] Tabla de relación N:M.
    \begin{itemize}
        \item \texttt{ContractId (FK, int)}
        \item \texttt{EstablishmentId (FK, int)}
    \end{itemize}

    \item[User (Usuario):] Autenticación y roles.
    \begin{itemize}
        \item \texttt{Id (PK, int)}
        \item \texttt{Email (varchar)}
        \item \texttt{PasswordHash (varchar)}
        \item \texttt{PersonId (FK, int, nullable)}
    \end{itemize}

    \item[AuditLog (Auditoría):] Registro inmutable.
    \begin{itemize}
        \item \texttt{Id (PK, long)}
        \item \texttt{TableName (varchar)}
        \item \texttt{Action (varchar)}
        \item \texttt{OldValues (jsonb)}
        \item \texttt{NewValues (jsonb)}
        \item \texttt{ChangedBy (varchar)}
        \item \texttt{ChangedAt (datetime)}
    \end{itemize}

    \item[PaymentTransaction (Transacciones):] Procesos financieros.
    \begin{itemize}
        \item \texttt{Id (PK, guid)}
        \item \texttt{ContractId (FK, int)}
        \item \texttt{Amount (decimal)}
        \item \texttt{Status (varchar)}
        \item \texttt{ProviderReference (varchar)}
        \item \texttt{PaymentDate (datetime)}
    \end{itemize}
\end{description}

%--------------------------------------------------------------------

\section{Especificación de API REST (Ejemplos)}
\label{app:api_docs}

\subsection{Creación de Contrato (POST /api/contract)}

\begin{lstlisting}[language=json, caption={Payload JSON para la creación de un contrato.}, label={lst:json_contract_create}]
{
  "personId": 1054,
  "startDate": "2023-01-01T00:00:00Z",
  "endDate": "2023-12-31T23:59:59Z",
  "establishmentIds": [ 12, 14 ],
  "clauseIds": [ 1, 2, 5 ],
  "observations": "Contrato renovado con adición presupuestal."
}
\end{lstlisting}

\subsection{Respuesta Exitosa (201 Created)}

\begin{lstlisting}[language=json, caption={Respuesta JSON tras la creación exitosa.}, label={lst:json_contract_response}]
{
  "id": 2045,
  "fullName": "Juan Perez",
  "totalBaseRentAgreed": 1500000.00,
  "active": true,
  "createdAt": "2023-10-27T10:30:00Z"
}
\end{lstlisting}

%--------------------------------------------------------------------

\section{Estructura del Proyecto (.NET Solution)}
\label{app:project_structure}

\begin{verbatim}
GESCOMPH.sln
|-- GESCOMPH.Entity (Core)
|   |-- Domain
|   |   |-- Models
|   |   |   |-- Contract.cs
|   |   |   |-- Establishment.cs
|   |-- DTOs
|       |-- ContractCreateDto.cs
|
|-- GESCOMPH.Business (Logic)
|   |-- Interfaces
|   |   |-- IContractService.cs
|   |-- Services
|       |-- ContractService.cs
|       |-- TokenBusiness.cs
|
|-- GESCOMPH.Data (Infrastructure)
|   |-- Context
|   |   |-- ApplicationDbContext.cs
|   |-- Services
|       |-- ContractRepository.cs
|
|-- GESCOMPH.WebGESCOMPH (Presentation)
    |-- Controllers
    |   |-- ContractController.cs
    |-- Program.cs
    |-- appsettings.json
\end{verbatim}

%--------------------------------------------------------------------

\section{Configuración del Entorno}
\label{app:env_vars}

\begin{table}[H]
    \centering
    \caption{Variables de Entorno Críticas}
    \label{tab:env_vars}
    \resizebox{\linewidth}{!}{%
        \begin{tabular}{@{}lll@{}}
        \toprule
        \textbf{Variable} & \textbf{Descripción} & \textbf{Ejemplo (Oculto)} \\ \midrule
        \texttt{ConnectionStrings\_\_Default} & Cadena de conexión SQL Server & \texttt{Server=db;Database=ges...} \\
        \texttt{JwtSettings\_\_Key} & Clave secreta para tokens & \texttt{[REDACTED\_256\_BIT\_KEY]} \\
        \texttt{JwtSettings\_\_Issuer} & Emisor del token & \texttt{gescomph-api} \\
        \texttt{MercadoPago\_\_AccessToken} & Token de pagos & \texttt{TEST-8493...} \\ 
        \texttt{ASPNETCORE\_ENVIRONMENT} & Entorno de ejecución & \texttt{Production} \\ \bottomrule
        \end{tabular}
    }% ← cierre correcto de resizebox
\end{table}
