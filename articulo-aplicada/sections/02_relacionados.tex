\section{Marco teórico y trabajos relacionados}
\label{sec:relacionados}

La ingeniería de software vive en un péndulo constante. Hace unos años, todo era consolidar; hoy, la norma parece ser distribuir. Sin embargo, adoptar microservicios solo porque "es lo moderno" es una receta para el desastre. En esta sección, no vamos a hacer un listado aburrido de papers; vamos a analizar qué dice realmente la evidencia sobre cuándo y cómo conviene romper un sistema, y cómo eso justifica la arquitectura de GESCOMPH.

\subsection{El dilema: ¿Monolito o Microservicios?}
Es fácil dejarse llevar por el éxito de gigantes como Netflix, cuyo viaje hacia los microservicios está muy bien documentado \cite{architectural_evolution_netflix_patlolla}. Pero hay que leer la letra pequeña. Mazzara et al. \cite{migration_monolith_mazzara} y Blinowski \cite{monolithic_vs_microservices_blinowski} ponen los pies en la tierra: si bien ganas escalabilidad, el precio a pagar en complejidad operativa es altísimo. No es una mejora automática; es un intercambio de problemas.

\begin{figure}[H]
    \centering
    \includegraphics[width=0.8\linewidth]{graphics/interfaz_gescomph.jpg}
    \caption{Comparativa conceptual entre Monolito y Microservicios. Se ilustra la complejidad operativa frente a la simplicidad de despliegue.}
    \label{fig:concepto_monolito_vs_micro}
\end{figure}

Para un proyecto académico como GESCOMPH, diseñado para explorar arquitecturas pragmáticas en contextos gubernamentales, la literatura sugiere prudencia. Velepucha y Flores \cite{survey_microservices_velepucha} advierten sobre los desafíos de migrar sin una estrategia clara. Por eso, nuestra apuesta por un diseño modular no es timidez, es una decisión pedagógica: demostrar que se puede preparar el terreno para escalar sin adoptar prematuramente la complejidad de sistemas distribuidos.

\subsection{No es solo código, es diseño}
De nada sirve tener microservicios si por dentro son un desastre. Nivedhaa \cite{software_architecture_evolution_nivedhaa} y Oyeniran \cite{microservices_cloud_native_oyeniran} recalcan que la arquitectura debe evolucionar con el negocio, no al revés. Aquí es donde brilla el \textit{Domain-Driven Design} (DDD). Myllynen et al. \cite{developing_conceptual_model_myllynen} muestran cómo modelar los límites del sistema basándose en el dominio real y no en tablas de base de datos es lo que realmente permite desacoplar componentes.

En GESCOMPH, tomamos esto muy en serio. No separamos clases por capricho técnico, sino siguiendo los límites del negocio de gestión contractual.

\begin{figure}[H]
    \centering
    \includegraphics[width=0.8\linewidth]{graphics/interfaz_gescomph_3.jpg}
    \caption{Modelado de dominios en arquitecturas modernas. La estructura del código debe reflejar los límites del negocio (DDD).}
    \label{fig:concepto_ddd}
\end{figure}

\subsection{Seguridad: Más piezas, más problemas}
Cuando rompes un monolito, multiplicas tu superficie de ataque. De repente, llamadas que eran internas ahora viajan por la red. Billawa et al. \cite{sok_security_billawa} lo dejan claro: la seguridad en sistemas distribuidos no puede ser un pensamiento secundario. Requiere una gestión de identidad robusta y observabilidad total.

No basta con poner un firewall. Kothapalli \cite{securing_microservices_kothapalli} y Cruz-Cunha \cite{security_microservices_cruz_cunha} insisten en la defensa en profundidad. Incluso se habla ya de "resiliencia cibernética predictiva" usando IA \cite{predictive_cyber_resilience_konakanchi}, aunque para nuestro alcance actual, nos centramos en lo fundamental y efectivo: una implementación sólida de JWT y RBAC que no deje brechas, siguiendo las recomendaciones de Patlolla \cite{enhancing_security_patlolla} sobre monitoreo constante.

\begin{figure}[H]
    \centering
    \includegraphics[width=0.8\linewidth]{graphics/interfaz_gescomph_2.jpg}
    \caption{Desafíos de seguridad en entornos distribuidos. La superficie de ataque aumenta exponencialmente con la fragmentación de servicios.}
    \label{fig:concepto_seguridad}
\end{figure}

\subsection{La deuda técnica que no se ve}
A veces, la peor deuda no es el código feo, es la mala arquitectura. Toledo et al. \cite{architectural_debt_toledo} explican cómo las decisiones tomadas bajo presión se calcifican y hacen que el sistema sea imposible de mantener años después.

GESCOMPH nace con la obsesión de evitar esto. Al usar \textit{Clean Architecture}, estamos pagando una "prima de seguro" inicial: escribimos más código ahora y separamos más capas de las que parecen necesarias, para que en el futuro, cuando cambien los requisitos (y cambiarán), el sistema no colapse bajo su propio peso.

\begin{figure}[H]
    \centering
    \includegraphics[width=0.8\linewidth]{graphics/interfaz_gescomph_4.jpg}
    \caption{Impacto de la deuda técnica arquitectónica. Las decisiones tempranas de diseño determinan la mantenibilidad a largo plazo.}
    \label{fig:concepto_deuda_tecnica}
\end{figure}