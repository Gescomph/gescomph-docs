Cualquiera que haya desarrollado software para el sector público conoce bien el problema: sistemas obsoletos, deuda técnica por todos lados y la sensación de que siempre estás apagando incendios. Con GESCOMPH queríamos romper ese ciclo. Este artículo presenta un sistema de gestión contractual que va contra la corriente de moda: no usamos microservicios. En su lugar, construimos un Monolito Modular con .NET 8 y SQL Server, algo que muchos dirían que está pasado de moda. Pero funcionó. La arquitectura nos dio lo mejor de ambos mundos: podemos desplegar todo de un tirón (nada de coordinar 20 servicios diferentes) y al mismo tiempo mantener los módulos bien separados gracias a Clean Architecture. A lo largo del artículo explicamos cómo implementamos la seguridad, incluyendo un sistema de rotación de tokens que nos dio más de un dolor de cabeza, las optimizaciones que hicimos con Entity Framework para no matar la base de datos, y cómo logramos que los costos en la nube no se disparen. Al final, los números nos dieron la razón: se puede tener un sistema mantenible sin caer en la complejidad innecesaria de los sistemas distribuidos. Creemos que esto puede servir como referencia práctica para otros proyectos de modernización en el gobierno.